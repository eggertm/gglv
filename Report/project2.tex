\documentclass[a4paper]{article}

\usepackage[english]{babel}
\usepackage[utf8]{inputenc}
\usepackage{float}
\usepackage{amsmath}
\usepackage{graphicx}
\usepackage{csquotes}
\usepackage{dirtytalk}
\usepackage{hyperref}
\usepackage{xcolor}
\usepackage{listings}
\lstset{
  basicstyle=\ttfamily,
  columns=fullflexible,
  frame=single,
  breaklines=true,
  postbreak=\mbox{\textcolor{red}{$\hookrightarrow$}\space},
}
\newcommand\todo[1]{\textcolor{red}{TODO: #1}}

\title{REI602M Machine Learning: Allstate Claims Severity}

\author{Eggert Jon Magnusson and Thor Tomasarson}

\date{\today}

\begin{document}
\maketitle

\section{Introduction}
\todo{...}

% 0.5 til 1 bls.
% * Hvað var gert?
% * Hvers vegna?
% * Hvernig?

\section{Implementation}
\todo{...}

% * Lýsing á aðferðafræði.
% * Stutt lýsing á þeim reikniritum sem eru notuð: flokkarar, aðhvarfsgreiningarlíkön, t-SNE, þyrpingagreining, …
% * Stutt lýsing á gagnasafni/söfnum.
% * Hvernig nákvæmni spálíkana er metin.
% * Hvernig gildi á “hyperparametrum” eru valin
%
% Nákvæmni í lýsingu á að miðast við að samnemendur ykkar í REI602M geti endurtekið tilraunirnar án teljandi vandræða

\section{Results}
\todo{...}

% * Fyrir verkefni #1 og #2: Töflur/myndir sem sýna t.d.
%   * Tíðnirit (e. histograms) fyrir stakar inntaksbreytur/úttaksbreytu, myndir sem sýna fylgni 2ja inntaksbreyta ofl.
%   * Nákvæmni einstakra líkana.
%   * Áhrif “hyperparametra” á gæði líkana.
%   * Yfirlit yfir inntaksbreytur sem mestu máli skipta.

% * Skrifið stutta lýsingu á því sem myndir og töflur sýna. Númerið allar myndir og töflur og notið númer þegar vísað er í úr texta (“Á mynd 2 má sjá …”) Gætið þess texti á myndum sé læsilegur (ekki of lítill).

% * Lýsið stuttlega tilraunum sem skiluðu litlu (“misheppnuðust”)

\section{Conclusion}
\todo{...}

% * Helstu ályktanir.
% * Næstu skref. (Hvernig mynduð þið halda áfram með verkefnið?)

\newpage
\appendix
\section{Appendix}

\subsection{Code}
\label{sec:code}
\lstinputlisting[caption={Python code}, label={lst:code}, language=Python, basicstyle=\small]{../Script.py}

\label{sec:output}
\lstinputlisting[caption={The output from execution of the python code}, label={lst:output}, basicstyle=\small, columns=flexible]{../ScriptOutput.txt}

% For example, you can cite the Nano 3 Lecture notes \cite{nano3}.
% \newpage
% \begin{thebibliography}{9}
% \bibitem{nano3}
%   K. Grove-Rasmussen og Jesper Nygård,
%   \emph{Kvantefænomener i Nanosystemer}.
%   Niels Bohr Institute \& Nano-Science Center, Københavns Universitet
% \end{thebibliography}
\end{document}